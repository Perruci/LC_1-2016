%%%%%%%%%%%%%%%%%%%%%%%%%%%%%%%%%%%%%%%%%
% University/School Laboratory Report
% LaTeX Template
% Version 3.1 (25/3/14)
%
% This template has been downloaded from:
% http://www.LaTeXTemplates.com
%
% Original author:
% Linux and Unix Users Group at Virginia Tech Wiki 
% (https://vtluug.org/wiki/Example_LaTeX_chem_lab_report)
%
% License:
% CC BY-NC-SA 3.0 (http://creativecommons.org/licenses/by-nc-sa/3.0/)
%
%%%%%%%%%%%%%%%%%%%%%%%%%%%%%%%%%%%%%%%%%

%----------------------------------------------------------------------------------------
%	PACKAGES AND DOCUMENT CONFIGURATIONS
%----------------------------------------------------------------------------------------

\documentclass{article}

\usepackage[utf8]{inputenc}
\usepackage[version=3]{mhchem} % Package for chemical equation typesetting
\usepackage{siunitx} % Provides the \SI{}{} and \si{} command for typesetting SI units
\usepackage{graphicx} % Required for the inclusion of images
\usepackage{natbib} % Required to change bibliography style to APA
\usepackage{amsmath} % Required for some math elements 
\usepackage{geometry}
 \geometry{
 a4paper,
 total={210mm,297mm},
 left=20mm,
 right=20mm,
 top=20mm,
 bottom=20mm,
 }
\setlength\parindent{0pt} % Removes all indentation from paragraphs

\renewcommand{\labelenumi}{\alph{enumi}.} % Make numbering in the enumerate environment by letter rather than number (e.g. section 6)

%\usepackage{times} % Uncomment to use the Times New Roman font

%----------------------------------------------------------------------------------------
%	DOCUMENT INFORMATION
%----------------------------------------------------------------------------------------

\newcommand{\hmwkTitle}{\ Formalizacao de Unicidade de Respostas para Algoritmos de Ordenacao } % Assignment title
\newcommand{\hmwkDueDate}{Domingo,\ 19\ de Junho\ 2016} % Due date
\newcommand{\hmwkClass}{Logica Computacional 1\ Turma D} % Course/class
\newcommand{\hmwkClassTime}{} % Class/lecture time
\newcommand{\hmwkClassInstructor}{Flavio L. C. Moura} % Teacher/lecturer
\newcommand{\hmwkAuthorName}{Lucas M. Chagas (12/0126643) e Pedro Henrique S. Perruci (14/0158596)} % Your name

% If you wish to include an abstract, uncomment the lines below
% \begin{abstract}
% Abstract text
% \end{abstract}

%----------------------------------------------------------------------------------------
%	SECTION 1
%----------------------------------------------------------------------------------------

\section{Introdução}

Neste projeto utilizamos o assistente de demonstração PVS para formalizar propriedades da unicidade de respostas para algoritmos de ordenação. 
Nos foi dado os arquivos unicity.pvs, unicity.prf, sorting.pvs, sorting.prf, sorting_seq.pvs e sorting_seq.prf, que continham as quatro quetões a serem formalizadas nesse projeto e outras propriedades auxiliares já formalizadas.


%----------------------------------------------------------------------------------------
%	SECTION 2
%----------------------------------------------------------------------------------------

\section{Questão 1}

Nesta questão, provamos que o primeiro elemento de uma lista sorteada  é menor ou igual a qualquer elemento da lista:
\\
\\


%----------------------------------------------------------------------------------------
%	SECTION 3
%----------------------------------------------------------------------------------------

\section{Questão 2}

Nesta questão, provamos que quick\_sort e bubblesort são funcionalmente equivalentes, utilizando os principais resultados da unicidade :\\
\\
\\
\textit{
func\_eqQB :  CONJECTURE\\
FORALL (l: list[nat]) :\\
 quick\_sort(l) = bubblesort(l)\\
}
\\
Para formalizar, usamos os lemas "quick\_sort\_works",\\
"bubblesort\_works" e "unicity\_sorted\_lists".\\
\\
Instanciamos na fórmula -1 quick\_sort(l) e bubblesort(l) para verificar a permutação \\
entre quick\_sort(l) e bubblesort(l) e verificar se as listas são sorteadas.\\

Primeiramente, dividimos a conjunção em 4 subárvores. 3 dos 4 itens são triviais. \\
\\
No caso onde não é trivial, tivemos que provar que as permutações de l, bubblesort(l) e quick\_sort(l), l são transitivas. \\
\\

Para provar isso, tivemos que chamar o lemma "permutations\_is\_transitive".\\


Instanciamos na fórmula -1 quick\_sort(l), l e bubblesort(l), onde chegamos em três subárvores, todas sendo trivias, completando a prova. \\

\\

%----------------------------------------------------------------------------------------
%	SECTION 4
%----------------------------------------------------------------------------------------

\section{Questão 3}

Na questão 3, assim como na questão 2, provamos que maxsort e heapsort são funcionalmente equivalentes, utilizando os principais resultados da unicidade :\\
\\
\textit{
	func\_eqQB :  CONJECTURE\\
	FORALL (s: finite_sequence[nat]) :\\
	maxosrt(s) = heapsort(s)\\
}
\\
Para formalizar, usamos os lemas "maxsort\_works",\\
"heapsort\_works" e "unicity\_sorted\_seqs".\\
Instanciamos na fórmula -1 maxsort(s) e heapsort(s) para verificar a permutação \\
entre maxsort(s) e heapsort(s) e verificar se as listas são sorteadas.
\\
Primeiramente, dividimos a conjunção em 4 subárvores. 3 dos 4 itens são triviais. \\
\\
No caso onde não é trivial, tivemos que provar que as permutações de s, maxsort(s) e s, heapsort(s) são transitivas e que a permutação(s, maxsort(s)) e permutação (maxsort(s),s) é simétrico. \\
\\

Para provar isso, tivemos que chamar o lemma "permutations\_equiv".\\
\\

Tivemos que expandir a fórmula de symmetric para verificar a simetria da permutação s, maxsort(s). Obtemos, assim, a simetria da permutação(s, maxsort(s)) e permutação(maxsort(s), s) \\
\\

Também expandimos a fórmula de transitive para verificar a transitividade entre a permutação(maxsort(s), s) e permutação (s, heapsort(s)).\\
Com a expansão, dividimos a conjunção e assim chegamos na conclusão da prova.\\
\\ 




%----------------------------------------------------------------------------------------
%	SECTION 5
%----------------------------------------------------------------------------------------

\section{Questão 4}

A questão pedia para provarmos que a unicidade entre as funções quick\_sort e maxsort :\\
\\
%----------------------------------------------------------------------------------------
%	SECTION 6
%----------------------------------------------------------------------------------------

\section{Conclusão}


%----------------------------------------------------------------------------------------
%	BIBLIOGRAPHY
%----------------------------------------------------------------------------------------


%----------------------------------------------------------------------------------------


\end{document}